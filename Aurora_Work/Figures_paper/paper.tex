\documentclass[12pt]{article}

\usepackage[utf8]{inputenc}
\usepackage{latexsym,amsfonts,amssymb,amsthm,amsmath, graphicx}
\usepackage[margin={0.75in,1in}]{geometry}
\usepackage{hyperref}


\title{Auroral Spirals Paper }
\author{Cassandra Litwinowich}

\begin{document}

\maketitle

\begin{center}
\section{Abstract}
\end{center}
In this case study, we investigate a close 
conjunction between auroral spirals observed by 
multi-wavelength all-sky auroral imagers on the ground
(from the TREx-RGB array) and the overflight of the Swarm B
satellite in the local morning sector. Due to an extended period
of low magnitude southward interplanetary magnetic field (IMF)
, a strong global convection pattern arises because of solar
wind-magnetosphere coupling. We find during the interval of the
conjunction, small-scale auroral spirals (∼100km length) form
coincident with a latitudinally narrow shear flow boundary 
measured by Swarm B. Additionally, measurements of the phase
speed of optical elements of the evolving auroral vortices
are consistent with the shear-flows seen by Swarm-B when
mapped into the field of view of the all-sky imager.
We further contextualize the conditions leading to the
auroral vortices by integrating a variety of local
electrodynamic and magnetic measurements into a 
meso-scale electrodynamics model. We find that the
location of the shear flow in the return convective
flow in the global pattern is coincident with the latitude of
the auroral vortices. These factors strongly suggest that a 
magnetic Kelvin-Helmholtz instability occurs on the inner edge 
of the dawnside convection cell, and that this shear-flow 
boundary provides the mechanism that drives the local auroral 
vortices seen in the ionosphere. Additionally, morphological 
arguments are preliminarily explained to explain folds coincident
with auroral spirals.
\newpage



\end{document}
